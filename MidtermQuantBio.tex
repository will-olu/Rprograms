% Options for packages loaded elsewhere
\PassOptionsToPackage{unicode}{hyperref}
\PassOptionsToPackage{hyphens}{url}
%
\documentclass[
]{article}
\usepackage{lmodern}
\usepackage{amssymb,amsmath}
\usepackage{ifxetex,ifluatex}
\ifnum 0\ifxetex 1\fi\ifluatex 1\fi=0 % if pdftex
  \usepackage[T1]{fontenc}
  \usepackage[utf8]{inputenc}
  \usepackage{textcomp} % provide euro and other symbols
\else % if luatex or xetex
  \usepackage{unicode-math}
  \defaultfontfeatures{Scale=MatchLowercase}
  \defaultfontfeatures[\rmfamily]{Ligatures=TeX,Scale=1}
\fi
% Use upquote if available, for straight quotes in verbatim environments
\IfFileExists{upquote.sty}{\usepackage{upquote}}{}
\IfFileExists{microtype.sty}{% use microtype if available
  \usepackage[]{microtype}
  \UseMicrotypeSet[protrusion]{basicmath} % disable protrusion for tt fonts
}{}
\makeatletter
\@ifundefined{KOMAClassName}{% if non-KOMA class
  \IfFileExists{parskip.sty}{%
    \usepackage{parskip}
  }{% else
    \setlength{\parindent}{0pt}
    \setlength{\parskip}{6pt plus 2pt minus 1pt}}
}{% if KOMA class
  \KOMAoptions{parskip=half}}
\makeatother
\usepackage{xcolor}
\IfFileExists{xurl.sty}{\usepackage{xurl}}{} % add URL line breaks if available
\IfFileExists{bookmark.sty}{\usepackage{bookmark}}{\usepackage{hyperref}}
\hypersetup{
  pdftitle={Midterm},
  pdfauthor={Ade Olu-Ajeigbe},
  hidelinks,
  pdfcreator={LaTeX via pandoc}}
\urlstyle{same} % disable monospaced font for URLs
\usepackage[margin=1in]{geometry}
\usepackage{color}
\usepackage{fancyvrb}
\newcommand{\VerbBar}{|}
\newcommand{\VERB}{\Verb[commandchars=\\\{\}]}
\DefineVerbatimEnvironment{Highlighting}{Verbatim}{commandchars=\\\{\}}
% Add ',fontsize=\small' for more characters per line
\usepackage{framed}
\definecolor{shadecolor}{RGB}{248,248,248}
\newenvironment{Shaded}{\begin{snugshade}}{\end{snugshade}}
\newcommand{\AlertTok}[1]{\textcolor[rgb]{0.94,0.16,0.16}{#1}}
\newcommand{\AnnotationTok}[1]{\textcolor[rgb]{0.56,0.35,0.01}{\textbf{\textit{#1}}}}
\newcommand{\AttributeTok}[1]{\textcolor[rgb]{0.77,0.63,0.00}{#1}}
\newcommand{\BaseNTok}[1]{\textcolor[rgb]{0.00,0.00,0.81}{#1}}
\newcommand{\BuiltInTok}[1]{#1}
\newcommand{\CharTok}[1]{\textcolor[rgb]{0.31,0.60,0.02}{#1}}
\newcommand{\CommentTok}[1]{\textcolor[rgb]{0.56,0.35,0.01}{\textit{#1}}}
\newcommand{\CommentVarTok}[1]{\textcolor[rgb]{0.56,0.35,0.01}{\textbf{\textit{#1}}}}
\newcommand{\ConstantTok}[1]{\textcolor[rgb]{0.00,0.00,0.00}{#1}}
\newcommand{\ControlFlowTok}[1]{\textcolor[rgb]{0.13,0.29,0.53}{\textbf{#1}}}
\newcommand{\DataTypeTok}[1]{\textcolor[rgb]{0.13,0.29,0.53}{#1}}
\newcommand{\DecValTok}[1]{\textcolor[rgb]{0.00,0.00,0.81}{#1}}
\newcommand{\DocumentationTok}[1]{\textcolor[rgb]{0.56,0.35,0.01}{\textbf{\textit{#1}}}}
\newcommand{\ErrorTok}[1]{\textcolor[rgb]{0.64,0.00,0.00}{\textbf{#1}}}
\newcommand{\ExtensionTok}[1]{#1}
\newcommand{\FloatTok}[1]{\textcolor[rgb]{0.00,0.00,0.81}{#1}}
\newcommand{\FunctionTok}[1]{\textcolor[rgb]{0.00,0.00,0.00}{#1}}
\newcommand{\ImportTok}[1]{#1}
\newcommand{\InformationTok}[1]{\textcolor[rgb]{0.56,0.35,0.01}{\textbf{\textit{#1}}}}
\newcommand{\KeywordTok}[1]{\textcolor[rgb]{0.13,0.29,0.53}{\textbf{#1}}}
\newcommand{\NormalTok}[1]{#1}
\newcommand{\OperatorTok}[1]{\textcolor[rgb]{0.81,0.36,0.00}{\textbf{#1}}}
\newcommand{\OtherTok}[1]{\textcolor[rgb]{0.56,0.35,0.01}{#1}}
\newcommand{\PreprocessorTok}[1]{\textcolor[rgb]{0.56,0.35,0.01}{\textit{#1}}}
\newcommand{\RegionMarkerTok}[1]{#1}
\newcommand{\SpecialCharTok}[1]{\textcolor[rgb]{0.00,0.00,0.00}{#1}}
\newcommand{\SpecialStringTok}[1]{\textcolor[rgb]{0.31,0.60,0.02}{#1}}
\newcommand{\StringTok}[1]{\textcolor[rgb]{0.31,0.60,0.02}{#1}}
\newcommand{\VariableTok}[1]{\textcolor[rgb]{0.00,0.00,0.00}{#1}}
\newcommand{\VerbatimStringTok}[1]{\textcolor[rgb]{0.31,0.60,0.02}{#1}}
\newcommand{\WarningTok}[1]{\textcolor[rgb]{0.56,0.35,0.01}{\textbf{\textit{#1}}}}
\usepackage{graphicx,grffile}
\makeatletter
\def\maxwidth{\ifdim\Gin@nat@width>\linewidth\linewidth\else\Gin@nat@width\fi}
\def\maxheight{\ifdim\Gin@nat@height>\textheight\textheight\else\Gin@nat@height\fi}
\makeatother
% Scale images if necessary, so that they will not overflow the page
% margins by default, and it is still possible to overwrite the defaults
% using explicit options in \includegraphics[width, height, ...]{}
\setkeys{Gin}{width=\maxwidth,height=\maxheight,keepaspectratio}
% Set default figure placement to htbp
\makeatletter
\def\fps@figure{htbp}
\makeatother
\setlength{\emergencystretch}{3em} % prevent overfull lines
\providecommand{\tightlist}{%
  \setlength{\itemsep}{0pt}\setlength{\parskip}{0pt}}
\setcounter{secnumdepth}{-\maxdimen} % remove section numbering

\title{Midterm}
\author{Ade Olu-Ajeigbe}
\date{3/19/2021}

\begin{document}
\maketitle

\hypertarget{r-markdown}{%
\subsection{R Markdown}\label{r-markdown}}

This is an R Markdown document. Markdown is a simple formatting syntax
for authoring HTML, PDF, and MS Word documents. For more details on
using R Markdown see \url{http://rmarkdown.rstudio.com}.

When you click the \textbf{Knit} button a document will be generated
that includes both content as well as the output of any embedded R code
chunks within the document. You can embed an R code chunk like this:

\begin{Shaded}
\begin{Highlighting}[]
\CommentTok{#This question has related parts.}
\CommentTok{#(a) (5 points) Based on your chosen 1000 bases create a transition matrix 𝑃.}
\CommentTok{#(b) (1+2 points) Verify if the rows of 𝑃 add up to 1. Why is that?}
\CommentTok{#(c) (7 points) Using this transition matrix 𝑃 and any initial probability vector 𝜋 of your choice (get creative here!), generate a Markov chain of length 5000.}
\CommentTok{#(d) (5 points) Repeat 1(b) for this chain you generated. Draw conclusions.}



\CommentTok{#transition matrix}
\NormalTok{P<-}\KeywordTok{prop.table}\NormalTok{(}\KeywordTok{table}\NormalTok{(ecoli[}\DecValTok{5000}\OperatorTok{:}\DecValTok{5999}\NormalTok{], ecoli[}\DecValTok{5001}\OperatorTok{:}\DecValTok{6000}\NormalTok{]),}\DecValTok{1}\NormalTok{)}
\NormalTok{P}
\end{Highlighting}
\end{Shaded}

\begin{verbatim}
##    
##             a         c         g         t
##   a 0.3308824 0.1911765 0.1617647 0.3161765
##   c 0.2938776 0.1836735 0.2938776 0.2285714
##   g 0.2489083 0.3362445 0.2401747 0.1746725
##   t 0.2086614 0.2795276 0.2322835 0.2795276
\end{verbatim}

\begin{Shaded}
\begin{Highlighting}[]
\CommentTok{#running a summation of the rows}
\CommentTok{#The rows equal 1 because each nucleotide is a proportion of the total sequence}
\KeywordTok{rowSums}\NormalTok{(P)}
\end{Highlighting}
\end{Shaded}

\begin{verbatim}
## a c g t 
## 1 1 1 1
\end{verbatim}

\begin{Shaded}
\begin{Highlighting}[]
\CommentTok{# C) taking the value of pi and transition matrix P }
\NormalTok{pi <-}\KeywordTok{c}\NormalTok{(}\DecValTok{1}\NormalTok{,}\DecValTok{2}\NormalTok{,}\DecValTok{1}\NormalTok{,}\DecValTok{2}\NormalTok{)}\OperatorTok{/}\DecValTok{4}

\CommentTok{#using the }
\CommentTok{#full nucleotide alphabet}
\NormalTok{nucleotides <-}\StringTok{ }\KeywordTok{c}\NormalTok{(}\StringTok{"a"}\NormalTok{,}\StringTok{"c"}\NormalTok{,}\StringTok{"t"}\NormalTok{,}\StringTok{"g"}\NormalTok{)}
\NormalTok{length<-}\DecValTok{5000}\CommentTok{# length of the}
\NormalTok{chain<-}\KeywordTok{rep}\NormalTok{(}\OtherTok{NA}\NormalTok{,length)}
\NormalTok{chain[}\DecValTok{1}\NormalTok{]<-}\KeywordTok{sample}\NormalTok{(nucleotides,}\DecValTok{1}\NormalTok{,}\DataTypeTok{p=}\NormalTok{pi)}
\NormalTok{chain[}\DecValTok{1}\NormalTok{]}
\end{Highlighting}
\end{Shaded}

\begin{verbatim}
## [1] "c"
\end{verbatim}

\begin{Shaded}
\begin{Highlighting}[]
\ControlFlowTok{for}\NormalTok{ (i }\ControlFlowTok{in} \DecValTok{1}\OperatorTok{:}\NormalTok{(length}\DecValTok{-1}\NormalTok{))\{ }
\NormalTok{ chain[i}\OperatorTok{+}\DecValTok{1}\NormalTok{]<-}\KeywordTok{sample}\NormalTok{(nucleotides,}\DecValTok{1}\NormalTok{,}\DataTypeTok{p=}\NormalTok{P[chain[i],])}
\NormalTok{ \}}
\KeywordTok{head}\NormalTok{(chain)}
\end{Highlighting}
\end{Shaded}

\begin{verbatim}
## [1] "c" "a" "c" "t" "g" "a"
\end{verbatim}

\begin{Shaded}
\begin{Highlighting}[]
\NormalTok{xsx <-P}\OperatorTok\NormalTok{P}
\NormalTok{xsx}
\end{Highlighting}
\end{Shaded}

\begin{verbatim}
##    
##             a         c         g         t
##   a 0.2719040 0.2411435 0.2220019 0.2649506
##   c 0.2720590 0.2526252 0.2251918 0.2501240
##   g 0.2774030 0.2389278 0.2373367 0.2463325
##   t 0.2673332 0.2474727 0.2366192 0.2485750
\end{verbatim}

\begin{Shaded}
\begin{Highlighting}[]
\CommentTok{#D row sums representing the new Markov chain}
\KeywordTok{rowSums}\NormalTok{(xsx)}
\end{Highlighting}
\end{Shaded}

\begin{verbatim}
## a c g t 
## 1 1 1 1
\end{verbatim}

\hypertarget{including-plots}{%
\subsection{Including Plots}\label{including-plots}}

You can also embed plots, for example:

\begin{verbatim}
##         c 
## 0.2448042
\end{verbatim}

\begin{verbatim}
## [1] "alpha score"
\end{verbatim}

\begin{verbatim}
## [1] 1.644854
\end{verbatim}

\begin{verbatim}
##          c          c 
## -0.1088155  0.5984239
\end{verbatim}

\begin{verbatim}
##      a 
## 499967
\end{verbatim}

\begin{verbatim}
##               [,1]         [,2]
## [1,] -5.631377e-07 4.563402e-06
## [2,] -5.631377e-07 4.563402e-06
## [3,] -5.631377e-07 4.563402e-06
## [4,] -5.631377e-07 4.563402e-06
## [5,] -5.631377e-07 4.563402e-06
## [6,] -5.631377e-07 4.563402e-06
\end{verbatim}

Note that the \texttt{echo\ =\ FALSE} parameter was added to the code
chunk to prevent printing of the R code that generated the plot.

\begin{Shaded}
\begin{Highlighting}[]
\CommentTok{#Suppose that we are interested in the restriction enzyme ‘GAATTC’. Use 𝑝𝐴, 𝑝𝐶, 𝑝𝑇 and 𝑝𝐺 as in the Ecoli dataset. [estimate them from the whole sequence]}
\CommentTok{#(a) (5 points) Let X denote the number of restriction sites in a sequence of length 5000. What is the probability distribution of X? Name the distribution and identify its parameters.}
\CommentTok{#(b) (5 points) What is the approximate distribution of X? Name the distribution and identify its parameters.}
\CommentTok{#(c) (5+5 points) In a sequence of length 5000, find the exact and approximate probability that X is at least 25.}

\CommentTok{#P(X<=5000)}

\CommentTok{#a)}

\NormalTok{tab<-}\KeywordTok{table}\NormalTok{(ecoli)[}\KeywordTok{c}\NormalTok{(}\StringTok{"a"}\NormalTok{,}\StringTok{"c"}\NormalTok{,}\StringTok{"t"}\NormalTok{,}\StringTok{"g"}\NormalTok{)]}
\NormalTok{tab}
\end{Highlighting}
\end{Shaded}

\begin{verbatim}
## ecoli
##       a       c       t       g 
## 1370126 1393804 1366560 1391841
\end{verbatim}

\begin{Shaded}
\begin{Highlighting}[]
\NormalTok{tabss<-}\StringTok{ }\KeywordTok{sum}\NormalTok{(tab[}\KeywordTok{c}\NormalTok{(}\StringTok{"a"}\NormalTok{,}\StringTok{"c"}\NormalTok{,}\StringTok{"t"}\NormalTok{,}\StringTok{"g"}\NormalTok{)])}
\NormalTok{tabss}
\end{Highlighting}
\end{Shaded}

\begin{verbatim}
## [1] 5522331
\end{verbatim}

\begin{Shaded}
\begin{Highlighting}[]
\NormalTok{proptabs<-tab[}\KeywordTok{c}\NormalTok{(}\StringTok{"a"}\NormalTok{,}\StringTok{"c"}\NormalTok{,}\StringTok{"t"}\NormalTok{,}\StringTok{"g"}\NormalTok{)]}\OperatorTok{/}\KeywordTok{sum}\NormalTok{(tab[}\KeywordTok{c}\NormalTok{(}\StringTok{"a"}\NormalTok{,}\StringTok{"c"}\NormalTok{,}\StringTok{"t"}\NormalTok{,}\StringTok{"g"}\NormalTok{)])}
\NormalTok{proptabs}
\end{Highlighting}
\end{Shaded}

\begin{verbatim}
## ecoli
##         a         c         t         g 
## 0.2481065 0.2523941 0.2474607 0.2520387
\end{verbatim}

\begin{Shaded}
\begin{Highlighting}[]
\NormalTok{restrictionenzyme<-}\KeywordTok{c}\NormalTok{(}\StringTok{"g"}\NormalTok{,}\StringTok{"a"}\NormalTok{,}\StringTok{"t"}\NormalTok{,}\StringTok{"t"}\NormalTok{, }\StringTok{"c"}\NormalTok{)}
\NormalTok{restrictionenzyme}
\end{Highlighting}
\end{Shaded}

\begin{verbatim}
## [1] "g" "a" "t" "t" "c"
\end{verbatim}

\begin{Shaded}
\begin{Highlighting}[]
\NormalTok{t}
\end{Highlighting}
\end{Shaded}

\begin{verbatim}
## function (x) 
## UseMethod("t")
## <bytecode: 0x00000000159099e8>
## <environment: namespace:base>
\end{verbatim}

\begin{Shaded}
\begin{Highlighting}[]
\CommentTok{#a)Normal Distribution Probability x = 5000, ecoli data = tabss X~(x<=5000, tabss, p = 0.10)}
 \KeywordTok{pnorm}\NormalTok{(}\DecValTok{5000}\NormalTok{, tabss, }\FloatTok{0.5}\NormalTok{)}
\end{Highlighting}
\end{Shaded}

\begin{verbatim}
## [1] 0
\end{verbatim}

\begin{Shaded}
\begin{Highlighting}[]
\CommentTok{#b)The length of a string of }
 \KeywordTok{pnorm}\NormalTok{(}\DecValTok{5000}\NormalTok{,(}\DecValTok{5000}\OperatorTok{*}\FloatTok{0.5}\NormalTok{),}\KeywordTok{sqrt}\NormalTok{(tabss}\OperatorTok{*}\FloatTok{0.5}\OperatorTok{*}\FloatTok{0.75}\NormalTok{))}
\end{Highlighting}
\end{Shaded}

\begin{verbatim}
## [1] 0.9588288
\end{verbatim}

\begin{Shaded}
\begin{Highlighting}[]
\CommentTok{#c)}
\KeywordTok{pnorm}\NormalTok{(}\FloatTok{25.5}\NormalTok{,(}\DecValTok{5000}\OperatorTok{*}\FloatTok{0.5}\NormalTok{),}\KeywordTok{sqrt}\NormalTok{(tabss}\OperatorTok{*}\FloatTok{0.5}\OperatorTok{*}\FloatTok{0.5}\NormalTok{))}
\end{Highlighting}
\end{Shaded}

\begin{verbatim}
## [1] 0.01760262
\end{verbatim}

\begin{Shaded}
\begin{Highlighting}[]
\CommentTok{#5. This question has two unrelated parts.}
\CommentTok{#(a) (10 points) A particular town has two retirement homes A and B. The number of persons vaccinated for COVID-19 in home A has a Binomial distribution with 𝑝 = 0.7. The number of persons vaccinated for COVID-19 in home B has a Binomial distribution with 𝑝 = 0.8. One of the homes is chosen at random (both homes are equally likely) and a sample of 25 is randomly selected. Given that 20 of the 25 persons selected were vaccinated, what is the probability that the sample was taken in home B?}
\CommentTok{#(b) (10 points) Let 𝑋 be a continuous random variable with pdf:}

\CommentTok{#Using functions and integration in R, compute 𝐸(𝑋) and 𝑉(𝑋).}

\CommentTok{##A ~ (n =25, p = 0.7)}
\CommentTok{#B<-( n = 25,p = 0.8)}
\CommentTok{#k = 20}

\CommentTok{#So given that is 0.8 x = 20, n = 25}
\CommentTok{#Exact probability}
\KeywordTok{pbinom}\NormalTok{(}\DecValTok{20}\NormalTok{,}\DecValTok{25}\NormalTok{, }\FloatTok{0.8}\NormalTok{) }
\end{Highlighting}
\end{Shaded}

\begin{verbatim}
## [1] 0.5793257
\end{verbatim}

\begin{Shaded}
\begin{Highlighting}[]
\KeywordTok{print}\NormalTok{(}\StringTok{"exact probability"}\NormalTok{)}
\end{Highlighting}
\end{Shaded}

\begin{verbatim}
## [1] "exact probability"
\end{verbatim}

\begin{Shaded}
\begin{Highlighting}[]
\NormalTok{pbinom }
\end{Highlighting}
\end{Shaded}

\begin{verbatim}
## function (q, size, prob, lower.tail = TRUE, log.p = FALSE) 
## .Call(C_pbinom, q, size, prob, lower.tail, log.p)
## <bytecode: 0x0000000012eaf680>
## <environment: namespace:stats>
\end{verbatim}

\begin{Shaded}
\begin{Highlighting}[]
\CommentTok{#Bin distribution approximation with the }
\KeywordTok{pnorm}\NormalTok{(}\DecValTok{20}\NormalTok{, (}\DecValTok{25}\OperatorTok{*}\FloatTok{0.8}\NormalTok{), }\KeywordTok{sqrt}\NormalTok{(}\DecValTok{25}\OperatorTok{*}\FloatTok{0.8}\OperatorTok{*}\FloatTok{0.2}\NormalTok{))}
\end{Highlighting}
\end{Shaded}

\begin{verbatim}
## [1] 0.5
\end{verbatim}

\begin{Shaded}
\begin{Highlighting}[]
\CommentTok{#probability form the binomial distribution using normal distribution to find the probability}
\NormalTok{xs <-}\KeywordTok{pnorm}\NormalTok{(}\FloatTok{20.5}\NormalTok{,(}\DecValTok{25}\OperatorTok{*}\FloatTok{0.8}\NormalTok{), }\KeywordTok{sqrt}\NormalTok{(}\DecValTok{25}\OperatorTok{*}\FloatTok{0.8}\OperatorTok{*}\FloatTok{0.2}\NormalTok{))}
\KeywordTok{print}\NormalTok{(}\StringTok{"continuity correction"}\NormalTok{)}
\end{Highlighting}
\end{Shaded}

\begin{verbatim}
## [1] "continuity correction"
\end{verbatim}

\begin{Shaded}
\begin{Highlighting}[]
\NormalTok{xs }
\end{Highlighting}
\end{Shaded}

\begin{verbatim}
## [1] 0.5987063
\end{verbatim}

\begin{Shaded}
\begin{Highlighting}[]
\CommentTok{#B}

\CommentTok{# take the integral((x/theta^2)(e))^(x^2/2*theta^2) = }
\NormalTok{theta =}\StringTok{ }\DecValTok{1}
\NormalTok{e =}\StringTok{ }\FloatTok{2.7}
\NormalTok{fex<-}\ControlFlowTok{function}\NormalTok{(x)\{(}\OperatorTok{-}\NormalTok{e}\OperatorTok{^}\NormalTok{(}\OperatorTok{-}\NormalTok{x}\OperatorTok{^}\DecValTok{2}\OperatorTok{/}\DecValTok{2}\OperatorTok{*}\NormalTok{theta}\OperatorTok{^}\DecValTok{2}\NormalTok{))}\OperatorTok{*}\KeywordTok{ifelse}\NormalTok{((x}\OperatorTok{>}\DecValTok{0} \OperatorTok{&}\StringTok{ }\NormalTok{x}\OperatorTok{<}\DecValTok{10000}\NormalTok{), }\DecValTok{1}\NormalTok{,}\DecValTok{0}\NormalTok{)\}}
\NormalTok{mew<-}\KeywordTok{integrate}\NormalTok{(fex,}\DecValTok{0}\NormalTok{,}\DecValTok{10000}\NormalTok{)}\OperatorTok{$}\NormalTok{value}
\NormalTok{mew}
\end{Highlighting}
\end{Shaded}

\begin{verbatim}
## [1] -1.101821e-24
\end{verbatim}

\begin{Shaded}
\begin{Highlighting}[]
\CommentTok{#V(X) = E(X-mew)^2}
\NormalTok{Xvar <-}\StringTok{ }\ControlFlowTok{function}\NormalTok{(x)\{(((x}\OperatorTok{-}\NormalTok{mew)}\OperatorTok{^}\DecValTok{2}\NormalTok{)}\OperatorTok{^}\NormalTok{(}\OperatorTok{-}\NormalTok{x}\OperatorTok{^}\DecValTok{2}\OperatorTok{/}\DecValTok{2}\OperatorTok{*}\NormalTok{theta}\OperatorTok{^}\DecValTok{2}\NormalTok{)) }\OperatorTok{*}\KeywordTok{ifelse}\NormalTok{((x}\OperatorTok{>}\StringTok{ }\DecValTok{0} \OperatorTok{&}\StringTok{ }\NormalTok{x}\OperatorTok{<}\StringTok{ }\DecValTok{10000}\NormalTok{), }\DecValTok{1}\NormalTok{, }\DecValTok{0}\NormalTok{)\}}
\NormalTok{nvar <-}\StringTok{ }\KeywordTok{integrate}\NormalTok{(Xvar,}\DecValTok{0}\NormalTok{,}\DecValTok{10000}\NormalTok{)}\OperatorTok{$}\NormalTok{value}
\NormalTok{nvar}
\end{Highlighting}
\end{Shaded}

\begin{verbatim}
## [1] 0
\end{verbatim}

\end{document}
